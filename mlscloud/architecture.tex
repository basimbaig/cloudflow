\section{Architecture \& Implementation}
\label{sec:architecture}

CloudFlow is a modular system composed of several asynchronously connected
components.  It provides a cloud-wide enforcement mechanism, assisting in
information flow control via active policy deployment and on demand VM
relocation.  CloudFlow only requires VM images and policies as input (Figure
\ref{cloudflow:figure:highlevelarch}).  Policies are propagated to nodes
on-demand at runtime. 
 
During execution, each VM has a set of application-specific labels
associated with its running tasks.  The appearance or disappearance of a
guest-hosted application with a given runtime label may trigger a
policy-defined action.  This action may be null or may result in the
relocation or halting of a subset of VMs.  The ultimate purpose is reactive
defense against any potentially unallowed information flows, as defined by the currently deployed policies.  

CloudFlow can naturally enforce different types of information flow policies
including Chinese Wall and Bell-Lapadula MLS models.  For example in the
case MLS policies, the runtime labels can be thought of as MLS
classification labels and CloudFlow as a runtime MLS enforcement mechanism.
\vspace{5pt}
\subsection{Introspection Module}
% 
CloudFlow deploys VM introspection to extract runtime labels
from key guest kernel data structures.  CloudFlow enforces information flow
in real-time which renders existing introspection libraries for KVM-QEMU
unusable due to their relatively high latency and low speeds.  For these reasons,
CloudFlow undertakes VM introspection by hooking into existing KVM-QEMU
internals and significantly reducing overheads due to unnecessary levels of
indirection added by external libraries.

Introspection is performed by a specialized introspection thread, one instance of which is placed near each monitored VM. 
The introspection thread runs as part of the KVM-QEMU hypervisor code base
\cite{QEMU} and exists inside the KVM-QEMU process.  It inserts itself inside the scheduling chain of hardware I/O threads
(within KVM-QEMU) that manage hardware I/O virtualization for the hypervisor
(Figure \ref{clowdflow:figure:introspectionmodule}).  This allows the use of
KVM-QEMU's internal mechanisms for quick access to guest memory. QEMU uses these mechanisms to manage DMA between I/O threads and the guest operating system however we leverage them for quick introspection. The
introspection thread synchronizes with the other threads using existing
locking mechanisms.  Yet since KVM-QEMU follows a hybrid approach that allows
both multi-threading and event based programming, the introspection
thread is also aware of internal scheduling demands of KVM-QEMU and yields
the I/O lock often enough to allow graceful handling of KVM-QEMU events.

With a hook into the KVM-QEMU's internal guest memory reading interface and
a number of synchronization issues resolved, the introspection thread feeds
from two key guest kernel data structures: the task list and the SELinux
hash table containing context information for all running tasks.  The thread
extracts this information periodically, monitoring for changes to the running
task list and associated SELinux context information.  The
thread receives a list of target labels of interest (from the policy
infrastructure) and continuously searches for them inside the target guest. 
Once a related change happens inside the guest kernel, the information is 
propagated from the guest to the policy checking modules within a few
milliseconds\footnote{This is a vast improvement over latency caused
by using existing introspection libraries as we show in Section
\ref{sec:evaluation}.} (which constitutes a vulnerability window discussed
later). 

Also, for extensibility and re-usability, the introspection thread is
segregated into two separate components.  The first component handles the
interfacing with KVM-QEMU internals, while the second component provides
the SELinux-aware data extraction mechanisms.

Since only the second component is entangled with CloudFlow particulars, the
first component in fact constitutes a new fast introspection interface for
KVM-QEMU which will be made public as a QEMU patch.

\subsection{Policy Language Model}

As previously mentioned, CloudFlow follows a policy-centric approach. Its
policy language provides a way to delineate information flow control
decisions in the form of high level policy constructs a subset of which is shown in Figure \ref{figure:policystructure}.

\begin{figure}[t]
\vspace{-5pt}
\begin{Verbatim}[frame=single]
<policy-list> = <policy>
	      | <policy> <policy-list>

<policy> =  <p> if <label> in <locality> 
                then <action> </p>

<label> = {Set of target runtime labels}

<locality> = <co-resident> | <self>

<action> = <migrate> | <halt> | resume
\end{Verbatim}
\vspace{0pt}
\caption{Subset of CloudFlow Policy Language}
\label{figure:policystructure}
\end{figure}
%add a footnote Obviously arbitrary Turing-complete language could be deployed here.
For simplicity and efficiency purposes we have decided to keep the CloudFlow language as simple as possible\footnote{Obviously any arbitrary Turing-complete language could be deployed here} while still serving as a sound illustrative example. Currently the policy language has a simple grammatical model. Each policy list is made up of one or more policies. Each policy expects a label, the
locality of the target virtual machine and an appropriate action to perform
in case of a positive match for the provided label. Presently, the set of labels is limited to
SELinux \cite{SeStat} types. However it must be noted that even though we use SELinux types as labels, our policies are simple in nature and do need expertise to craft as is the case with SELinux policies. We felt there is value in
coupling with guest SELinux ecosystems, to allow for a wide range of complex
policies, yet not require changes to existing guest logic. We note that the set of labels can be changed arbitrarily along with the underlying OS.

All label symbols form terminal symbols in the policy grammar. The set of localities is formed of two terminal symbols {\textless co-resident\textgreater} and {\textless self\textgreater}. As all policies are written from the viewpoint of a particular VM, this part of the policy outlines whether the label should be searched for in the VM itself (self) or in all the VMs co-resident to it (co-resident). Finally the set of actions is formed of three terminal symbols  {\textless migrate\textgreater}, {\textless halt\textgreater} and {\textless resume\textgreater}. These symbols are pretty self-explanatory as they outline the action that needs to be taken on the VM that contained the label specified by the current policy.

Consider for example
the case when VMs for users Alice and Bob need to be kept physically isolated
throughout the cloud infrastructure -- the corresponding policy for Bob is
illustrated in Figure \ref{cloudflow:figure:examplepolicy}. 
In this example, processes run by Alice are
marked as ``Alice'' (as well with a number of other labels, including the process
name) at runtime by the SELinux subsystem on each guest to facilitate in the introspection process.

Further, we have picked XML as the format of choice for the actual
implementation of the policy files.  The policy language itself
is agnostic to the format used in the implementation.  In fact, early versions of CloudFlow were using key-value pairs to implement the policy language.  XML has a flexible structure
that allows quick extensions with minimal effort. As CloudFlow policies
and associated scenarios mature we envision a set of graphical tools to
manipulate these cloud-wide policies. 

\begin{figure}[h]
%\vspace{0pt}
\begin{Verbatim}[frame=single]
<p> if <Alice> in <co-resident>
    then <migrate> </p>
\end{Verbatim}
\vspace{-10pt}
\caption{Example policy
\label{cloudflow:figure:examplepolicy}}
\end{figure}
\vspace{10pt}

\subsection{Management Module}

The CloudFlow management module is the component that provides an entry
point to the cloud administrator.  An administrator can input a VM image
along with a list of security policies written by a policy designer.  In addition to the usual management tasks,
such as deploying and tracking the VMs, the CloudFlow management module does all the
house-keeping such as storing and propagating
policies.  The module is integrated seamlessly in the OpenStack
platform \cite{OpenStack} and features additional code to handle call-backs from the
policy module as well as user interface events.

\subsection{Policy Module (Daemon, Master)}
% 
The CloudFlow management module propagates the provided policies to the
actual policy enforcement infrastructure.  CloudFlow runs a cloud-wide
policy master module (PolicyM) and a per-node instance of the policy daemon
(PolicyD).  PolicyM acts as a master that controls all running instances of
PolicyD and keeps track of conflicts across physical hosts.  Meanwhile,
PolicyD has the job of maintaining policy lists for a given physical node as
well as interfacing with the introspection module for each VM.

\begin{figure}[h]
\begin{center}
\includegraphics[scale=0.3]{figures/moduledesign2.eps}
%\vspace{30pt}
\caption{\small 
%
CloudFlow follows a completely modular design. VM management, policy
enforcement and VM introspection are all performed by dedicated modules.
%
\label{cloudflow:figure:moduledesign}}
\end{center}
\end{figure}

\begin{figure*}[t]
\begin{center}
\includegraphics[scale=0.70]{figures/illustrativepicture.eps}
%\vspace{30pt}
\caption{\small 
%
CloudFlow performs dynamic policy-triggered VM relocation based on input policies. 
An example policy is one that disallows two users (Alice and Bob) from being co-resident 
on the same physical node based on \textit{runtime labels}.
%
\label{cloudflow:figure:illustrativescenario}}
\end{center}
\end{figure*}

Whenever a new policy is passed to PolicyD by the management module at VM
deployment time, it parses target labels in the policy and passes them on to
the appropriate introspection module.  It also notifies PolicyM that a new
VM has become active in the cloud.  Similarly in the case of VM deletion,
PolicyM is informed by PolicyD of the change.  Additionally, whenever a
policy label ``matches'' a runtime label inside a guest VM, PolicyD gets
notified by the corresponding introspection module, in effect ``triggering''
a policy.  PolicyD looks up the required action as specified by this policy.  In case of
a ``halt'' action, PolicyD simply issues a halt call-back to the management
module.  In case of a ``migration'' it issues a request to PolicyM to search for
a potential migration target.

PolicyM maintains high level state for the entire cloud and in case of a
migration request from PolicyD, it scouts for a target node that does not
introduce additional conflicts.  To this end PolicyM executes a
reconnaissance phase to find the potential migration target.  It queries runtime
labels for all deployed VMs from the associated PolicyDs and crosschecks
them with deployed policies.  This allows for the selection of a migration
target that (at least in the immediate future) will not cause further
conflicts.  PolicyM then reports this host back to PolicyD, which issues the
migration request call-back to the management module.  The CloudFlow PolicyD and
PolicyM were written from scratch in Python and Java respectively.



\subsection{Illustrative Scenario}
%make sure illustrative diagram is here
Here we outline an illustrative scenario (Figure \ref{cloudflow:figure:illustrativescenario}). 
Consider a law firm with two clients A and B, handled by employees Alice and
Bob respectively.  State and federal laws prevent Alice and Bob from sharing
any information due to the possibility of a conflict of interest arising. 
In this scenario, CloudFlow prevents an information leakage as follows:




\begin{itemize}

\item A policy is created outlining the fact that workloads run by users
Alice and Bob should not be co-resident.  The simple policy in Figure
\ref{cloudflow:figure:examplepolicy} can be used as an example.  The policy
will be described for Alice and Bob and propagated automatically along with
both their VMs at time of deployment.

\item Sometime in the future Alice is working on client A's data and needs
to run her workload.

\item Bob uses the same cloud as Alice and it just so happens that the cloud
scheduler locates their VMs on the same physical machine {\em before they both
start their respective security-critical workloads}.

\item After some time Alice's VM boots up and launches the desired
Alice-labeled workload. 

\item The introspection thread assigned to Alice's VM will immediately raise a flag
as soon as it notices the Alice label on the applications.  This event is 
sent back to the PolicyD running on the current physical node.  PolicyD
looks at the policies that have been specified for the VM that issued the
call-back.  Although PolicyD finds one policy that specifies Alice, it knows
that currently there are no other VMs on the same machine running workloads
from Bob hence no action is taken.

\item After some more time passes, Bob's VM boots up and launches its
Bob-labeled workload.  At this point the introspection
thread for Bob's VM reports back to PolicyD
having seen the desired runtime label 'Bob'.

\item PolicyD now works out that an administrator has specified a policy
that prohibits the current cloud state (Alice and Bob have co-resident workloads).  Alice's VM is frozen to prevent 
any security breach.

\item The action specified by the policy i.e., migration, is then passed back
to the PolicyM module by PolicyD along with the id for Alice's VM.

\item PolicyM now begins a reconnaissance phase to locate a potential migration target
to relocate Alice's VM.  In this phase PolicyM queries a list of runtime
labels from all potential hosts and as soon as it finds a target host that
will not cause a policy conflict with Alice's VM, it returns the host name
to the PolicyD that issued the migration request.

\item PolicyD, having acquired a conflict-free migration target, issues a
migration request to the management module for relocating Alice's VM..

\item The OpenStack module issues a migrate action, as it normally would for
load balancing, and migrates Alice's VM to the provided migration target,
where the VM resumes.


\end{itemize}

