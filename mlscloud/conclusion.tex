\section{Conclusion and Future Work}
\label{sec:conclusion}

%\begin{enumerate}
%\item Summarize the problem we solved.
%\item Restate all our contributions that were mentioned in the introduction and say how our results really back up our contributions.
%\item Mention some future directions (that are non-trivial)
%\end{enumerate}

The main contributions of this paper include (i) a new introspection
mechanism for the KVM-QEMU \cite{QEMU} hypervisor that allows for fast
asynchronous virtual machine introspection at the hypervisor level, orders
of magnitude faster than previous approaches, and (ii) a cloud wide
information flow control layer for OpenStack \cite{OpenStack} that leverages
the introspection mechanisms to enforce best effort real-time information
flow control.


% We have restricted our runtime labels to conform to SELinux types and
% focused on the policy enforcement infrastructure rather than developing
% complicated label sets.  One exciting avenue for future work is enhancing
% CloudFlow to handle an arbitrary set of labels inside the guest VM.  The
% introspection module can be augmented with schemes that look at larger parts
% of the guest memory (as opposed to data structures used in this paper) and
% use this data to provide the same information flow control invariants for an
% environment where the guest operating system is potentially untrusted.

Ongoing and future work includes (a) new stronger introspection mechanisms
resilient to malicious guest OSes and (b) CloudFlow extensions that will
allow policies to be written in terms of additional guest state elements,
including I/O state, intra-OS events and IPC. 

