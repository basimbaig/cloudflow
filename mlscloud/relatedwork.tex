\section{Related Work}
\label{sec:relatedwork}
\paragraph{\bf Cloud Policy Enforcement}

The main focus of this paper is to provide information flow control at
cloud-level via policy enforcement.  There has been previous work in the
field of cloud wide policy enforcement but it is geared towards improving
performance rather than providing better security.  Policies are usually
defined in order to attain better segregation amongst nodes of different
computational and storage needs and provide better load balancing or
scheduling algorithms.  Work that most closely resembles this paper includes
stronger virtual machine isolation mechanisms \cite{Mushi}.  Such mechanisms
do not implement any policy mechanism and rather provide rudimentary MLS via
hardwired isolation.

\paragraph{\bf MLS}

MLS enforcement is an extensively studied field at the scope of an end host.
SELinux \cite{SeStat} is an established platform for providing MLS within
the operating system.  It was originally developed by the NSA but since then
it has been merged into the mainline Linux kernel due to widespread
adoption.  In this paper we have used the typing mechanisms provided by
SELinux for the purpose of labeling processes and users to generate runtime
labels.  However SELinux is not limited to this use and indeed it has been
used in a multitude of projects within the security world.  In addition to
this, works like HiStar \cite{HiStar} aim to provide a fully functioning
operating system with information flow monitoring.  This is a perfectly
viable approach but it places the burden of security on the developers
because existing applications need to be re-engineered from the ground up
and extensive changes are required to the existing hardware.  In our private
cloud scenario, we want to minimize the work needed to upgrade the existing
infrastructure which includes software, hardware and personnel.

\paragraph{\bf Side channel attacks}

Information leakage via side channels has been a well known issue for years.
Side channel attacks have been developed for a variety of systems and
communication channels and exploit timing information, electromagnetic leaks
and even power consumption.  However, in this paper we have narrowed our
scope to hardware side channels.  Generally speaking, side channel attacks
have been mounted in several scenarios to extract sensitive information. 
This includes cryptographic keys \cite{SideCrypto,SideCloud,PrimeProbe12} in the context
of operating systems as well as sensitive information on the web
\cite{LastSummer,ScriptlessAttacks}.  In the scope of cross-VM side channels
attacks, previous work \cite{HomeAlone} has been done to thwart cache based
attacks.  This is a perfectly viable approach but it remains powerless
against new attack vectors.  CloudFlow is driven by the philosophy that
hardware side-channels are arguably impossible to eliminate and thus the
best defense is preventive rather than curative.

\paragraph{\bf Virtual machine introspection}

Virtual machine introspection was first suggested by Garfield and Rosenblum
\cite{Terra}.  There are usually two types of introspection mechanisms,
those requiring changes to the guest virtual machine or target applications
and those that work at the hypervisor level.  In this paper we have opted
for the later approach to ensure minimum hassle in terms of changes needed
to existing operating system infrastructure.  External introspection is
extremely popular and has been used by a multitude of researchers
\cite{RemoteAttestation,OutOfBox,OutOfBoxMalware,AntFarm,VMIObserve,TamperResist,libvmi}. 
Previous introspection mechanisms have been aimed towards solving a myriad
of problems such as intrusion detection \cite{CoPilot}, malware analysis
\cite{Ether} and forensics.  However external introspection approaches are
inherently at a disadvantage due to having an outsider view and remain prone
to the \textit{semantic gap} problem \cite{SemanticGap}.

In recent years there has been much effort to reduce this so called semantic gap and provide detailed information even for adversarial virtual machines \cite{Virtuoso,ProcessImplanting}. This has led to construction of hybrid approaches that install an agent inside the monitored guest as well to provide a view that is richer in semantics and provides a detailed picture of the guest.

%we need to cite homealone here and in the intro as well
